%\documentclass[11pt,a4paper,uplatex,dvipdfmx]{ujarticle} 		% for uplatex
\documentclass[11pt,a4j,dvipdfmx]{jarticle} 					% for platex
\input{pieces/form00_header} % pieces
\input{pieces/kakenhi7} % pieces
\input{pieces/form01_dcpd_header} % pieces
\input{pieces/hook3} % pieces
%#Name: dcs
\input{pieces/form03_dcpd_headers} % pieces
\input{pieces/form04_dc_header} % pieces
% ===== Global definitions for the Kakenhi form ======================
% 基本情報
%
%------ 研究課題名  -------------------------------------------
\newcommand{\研究課題名}{COMET Phase-Iに向けたエンジニアリングランおよびその物理解析}

%----- 研究機関名と研究代表者の氏名-----------------------
\newcommand{\研究機関名}{大阪大学}
\newcommand{\研究代表者氏名}{高見 翔太   }
\newcommand{\me}{\underline{\underline{Shohta~Takami}}} 
\input{pieces/inst_general_images} % pieces
% user07_header
% ===== my favorite packages ====================================
% ここに、自分の使いたいパッケージを宣言して下さい。
\usepackage{ulem}
\usepackage{wrapfig}
\usepackage{amssymb}
%\usepackage{mb}
\usepackage{amsmath}
\usepackage{xcolor}
\usepackage{enumitem}
%\DeclareGraphicsRule{.tif}{png}{.png}{`convert #1 `dirname #1`/`basename #1 .tif`.png}
\usepackage{lineno}

% ===== my personal definitions ==================================
% ここに、自分のよく使う記号などを定義して下さい。
\renewcommand{\refname}{}%リファレンスの頭に「参考文献」っていうタイトルつけない
\newcommand{\klpionn}{K_L \to \pi^0 \nu \overline{\nu}}
\newcommand{\kppipnn}{K^+ \to \pi^+ \nu \overline{\nu}}
%\newcommand{\mysection}[2]{\hspace{-12pt}\colorbox[gray]{0.8}{\normalsize{\textbf{\ctext{#1}#2}}}\\}
\newcommand{\mysection}[2]{\hspace{-12pt}\colorbox{cyan!15}{\normalsize{\textbf{\ctext{#1}#2}}}\\}
\newcommand{\mysubsection}[1]{\vspace{-20pt}\subsection*{\colorbox{cyan!15}{\normalsize{#1}}}\vspace{-0.2cm}}
% ----- 業績リスト用 -------------
\newcommand{\paper}[6]{%
	% paper{title}{authors}{journal}{vol}{pages}{year}
	\item ``#1'', #2, #3 {\bf #4}, #5 (#6).			% お好みに合わせて変えてください。
}

\newcommand{\etal}{\textit{et al.\ }}
\newcommand{\ca}[1]{*#1}	% corresponding author;   \ca{\yukawa}  みたいにして使う
\newcommand{\invitedtalk}{招待講演}

\newcommand{\yukawa}{H.~Yukawa}					% no underline
%\newcommand{\yukawa}{\underline{\underline{H.~Yukawa}}}	% with 2 underlines
\newcommand{\tomonaga}{S.~Tomonaga}

\newcommand{\prl}{Phys.\ Rev.\ Lett.\ }		% よく使う雑誌も定義すると楽

% ===== 欄外メモ ==================
\newcommand{\memo}[1]{\marginpar{#1}}
%\renewcommand{\memo}[1]{}	% 全てのメモを表示させないようにするには、行頭の"%"を消す

%\input{../../sample/simple/contents}	% skip
\input{pieces/hook5} % pieces

\begin{document}

\input{pieces/hook7} % pieces
%#Split: 01_background  
%#PieceName: p01_background
\input{pieces/p01_background_00}
\section{研究の概要と位置づけ}
%    <<最大 1ページ>>

%s03_abst_background
\vspace{-0.2cm}
%\uwave{波線部はコメント修正箇所です}\\
\noindent
\underline{\textbf{研究課題名:\研究課題名}}

%begin 研究目的や背景などの概要 ====================
\vspace{0.1cm}
\mysubsection{概要}
%end 研究目的や背景などの概要 ====================

%begin 本研究の着想に至った経緯など ====================
素粒子物理学における標準模型は、物質の最小構成要素である\textbf{素粒子}における基本的な相互作用を記述する理論である。
標準模型は未だ不完全で、%\uwave{物質反物質非対称性やニュートリノ振動など説明できない現象が残っ}\\%\uwave{ているため}
新理論の発見が待ち望まれている。
本研究で探索する\textbf{荷電レプトンフレーバー非保存過程}(以下、charged Lepton Flavor Violation:cLFV)は
標準模型では崩壊分岐比が$10^{-54}$以下と非常に強く抑制されている。
一方、多くの新理論内ではこの過程が$10^{-15}$程度の確率で観測可能と示唆されている。
従って、cLFVの発見は標準模型の寄与がなく、\textbf{新物理の直接的な証拠}となる。
COMET実験ではミューオン電子転換過程($\mu^- + N \rightarrow e^- + N$)と呼ばれるcLFVの一種を探索する。
COMET実験は現状のミューオン電子転換過程に対する \uwave{世界最高感度を二桁ほど更新するものとなる}。
現在COMETでは\textbf{2026年末}に初の\textbf{ビーム供給}を控えており、
それに向けた検出器のインストールおよび物理解析に向けた較正などが\textbf{最優先の課題}として挙げられる。
\mysubsection{研究の位置付け}
\noindent \textbf{・当該分野の状況}\\
ミューオン電子転換探索の世界最高感度は、先行実験のSINDRUM-I\hspace{-1.2pt}Iでの$7 \times 10^{-13}$であり、発見には至らなかった$^{\cite{SINDRUM}}$。
この測定では連続的なビームを使用したため、ビーム由来の背景事象が常に存在し、ビー
\begin{wrapfigure}[9]{r}{190pt}{}
	\centering
	 \vspace{-0.5cm}
	 \hspace*{-15pt}
	 \includegraphics[width=7.5cm]{figs/beam}
	 \vspace{-1.2cm}
	 \caption{\small{ビームの時間構造}}
	 \label{Fig:beam}
	\end{wrapfigure} 
ム強度に対する探索感度に制限があった。
そこでCOMET実験では、パルス状のビームを使用することと標的にアルミニウムを使用することで背景事象の問題を克服する。
ビームのパルス間隔$1.17\;\rm{\mu s}$に対し、原子中のミューオン寿命が$0.86\;\rm{\mu s}$と長いアルミニウムをミューオン静止標的に採用し、陽子ビームのバンチタイミングから遅れてトリガーをかける。
これにより、ビーム由来のバックグラウンドが大幅に低減する。
同様の実験として米国Fermi研究所にて建設中のMu2e実験$^{\cite{Mu2e}}$があるが、COMET実験が先行して、世界最高感度に到達する計画である。\\
\noindent \textbf{・本研究の着想に至った経緯}\\
COMET実験における背景事象にはビーム由来のもの以外に、原子軌道上におけるミューオンの三体崩壊(Decay In Orbit:DIO)による電子があげられる。
そこで検出された電子の運動量を用いて、DIOによる電子とミューオン電子転換による電子を見分ける。
ミューオンビームのアルミニウム標的への照射は$1\;\rm{T}$のソレノイド磁場内で行う。
そのため、荷電粒子が磁場中で螺旋運動する性質を用いて、運動量を測定することができる。
具体的にはガス飛跡検出器(Cylindrical Drift Chamber:CDC)による電子の飛跡の曲率と磁場の値よりその運動量を測定することができる。
そこで、CDCで検出された背景事象を含むヒットのうち信号電子の飛跡を見つけるアルゴリズムの開発が必要不可欠であり、DIOによる信号と分離するためには$\mathbf{105\;\rm{\textbf{MeV/c}}}$の\textbf{運動量分解能}が要求されている。\\
 COMET実験は2026年に全ての装置のインストールが完了し、同年末には低強度ビームでの物理測定を開始し、2028年にはいよいよCOMET Phase-Iが開始する予定である。
従って2026年以降にはCDCの運動量較正や上述の物理解析手法の確立を行うことが優先順位の極めて高い項目である。
そこで、申請者の修士課程における\textbf{磁場分布に関する経験}やトラッキングアルゴリズムに対する理解を活かし、\textbf{世界最高感度でのミューオン電子転換過程探索}を実現する。

%\vspace{1cm}
\small
\vspace{-0.3cm}
\begin{thebibliography}{99}
	\bibitem{SINDRUM} SINDRUM-II Collaboration, Eur. Phys. J.C47 337-346 (2006)
	\vspace{-0.1cm}
	\bibitem{Mu2e}Mu2e Collaboration, Universe, 9, 54. (2023) 
\end{thebibliography}
%\begin{wrapfigure}[9]{r}{190pt}{}
%	\centering
%	 \vspace{-1.1cm}
%	 \hspace*{-15pt}
%	 \includegraphics[width=7cm]{figs/momentum.png}
%	 \vspace{-1.2cm}
%	 \caption{\small{運動量分布}}
%	 \label{Fig:beam}
%	\end{wrapfigure} 









%end 本研究の着想に至った経緯など ====================

\input{pieces/p01_background_01}

%#Split: 02_purpose_plan  
%#PieceName: p02_purpose_plan
% p02_purpose_plan_00.tex
\KLBeginSubjectWithHeaderCommands{02}{}{【2】研究計画(2)研究目的・内容等}{2}{F}{}{\DCPDFirstSubjectPageStyle}{\DCPDDefaultPageStyle}

\section{【2】研究計画(2)研究目的・内容等}
%    <<最大 2ページ>>
\mysubsection{1.研究目的、研究方法、研究内容}
本研究の目的はミューオン電子転換過程を世界最高感度で探索し、標準模型を破る新物理を実験的に観測することである。
そのため、茨城県の大強度陽子ビーム施設J-PARCで計画している\textbf{COMET実験}で測定および物理解析を主導する。\\
\begin{wrapfigure}[12]{r}{190pt}{}
	\centering
	 \vspace{-1.2cm}
	 \hspace*{-15pt}
	 \includegraphics[width=7cm]{figs/momentum.png}
	 \vspace{-1.0cm}
	 \caption{\small{信号電子の運動量分布}}
	 \label{Fig:momentum}
	\end{wrapfigure} 
 アルミニウム原子核に捕獲されたミューオンのミューオン電子転換によって生成される信号電子の運動量はミューオンの質量から原子核の束縛エネルギーと反跳エネルギーを引いた$\mathbf{105\;\rm{MeV/c}}$である。
一方、背景事象である原始軌道上でのミューオンの崩壊(DIO)は三体崩壊であるため、ミューオン電子転換過程よりも小さい運動量になる。
そこで検出器全体をソレノイド磁場内に設置し、粒子の飛跡より求められる\textbf{電子の螺旋運動の曲率}を用いて信号電子の運動量を測定する。
飛跡検出にはガス検出器(Cylindrical Drift Chamber:CDC)を用いる。
CDCの仕組みはガス中を通過した荷電粒子がガスをイオン化し、それによりできた電子イオン対が電場によりワイヤーに引き寄せられることで電気信号を得るというものである。
従って電場や磁場の情報から最適化した電子イオン対のドリフト時間をもとに反応したワイヤーの位置から荷電粒子の真のヒット位置を再構成する。
CDCのデータ取得のトリガーにはシンチレーター検出器CTHを用いる。
期待する螺旋運動にあわせて層状にシンチレーターを配置し、複数のシンチレーターを通過した時にCDCでデータ取得を行うようにトリガーをかける。
図\ref{Fig:momentum}のようにDIO電子と信号電子のスペクトルを分離するには$\mathbf{200\;\rm{keV/c}}$\textbf{の運動量分解能}がCDCには要求されている。\\
%COMET実験は2026年末から低強度ビームでの物理測定、2028年には150日間のビーム運転期間での物理データ測定が予定されている。
%まず、低強度ビームでの物理測定前に、通常のミューオンの崩壊(Mitchel Edge)や$\pi$の$e$への崩壊など標準模型内での電子への崩壊時過程を用いて、検出器の運動量較正を行う。
DIOや宇宙線由来の背景事象を効率よくカットし、\mbox{SINDRUM-I\hspace{-1.2pt}I$^{\cite{SINDRUM}}$}の感度を100倍更新する$\mathbf{3.0 \times 10^{-15}}$での\textbf{ミューオン電子転換過程探索}を実現する。\\


% 信号電子が検出器内で複数周回転するmultiple turn tracksが現在物理解析において問題視されている。その理由は三つあり、第一に、複数周CDCにヒットを残すため、周回ごとにヒットを区別する必要があり、トラッキング精度が悪くなってしまう。
%第二に、CDCでのヒットが多い分、失う運動量も大きくなり、DIOとの区別が難しくなる。
%第三に、周回が多い分、無関係なノイズを拾いやすい。
%結果的に検出器内で一周しかしないsingle turn tracksのアクセプタンスは$11\;\rm{\%}$であるのに対し、multiple turn tracksでは$7.2\;\rm{\%}$になる。
%つまり、multiple turn trackを判別できないと、最悪の場合COMET実験で期待できる\textbf{最高感度の2/3}ほどしか達成できない可能性がある。
%そこで、multiple turn tracksを判別できる解析法を開発し、\mbox{SINDRUM-I\hspace{-1.2pt}I$^{\cite{SINDRUM}}$}の感度を100倍更新する$\mathbf{3.0 \times 10^{-15}}$での\textbf{ミューオン電子転換過程探索}を実現する。\\
\vspace{-0.5cm}
\mysubsection{2.研究計画}

%\noindent \textbf{【磁場分布の運動量スペクトルへの寄与の評価(採用まで)】}\\
% 信号電子の運動量測定には磁場分布の情報が必要不可欠である。
%申請者は修士課程ではCOMET Phase-Iにおける検出器ソレノイド(Detector Solenoid:DS)の磁場測定に携わっており、
%本番のビームエリアに
%物理ランを行う時のセットアップと同じ本番環境における磁場測定システムの設計には
%運動量分解能$\mathbf{200\;\rm{keV/c}}$を達成するために磁場測定の精度の要求値を決定する必要がある。
%そのために、磁場測定の精度の情報をCOMET実験グループ内で独自に開発されたICEDUST$^{\cite{ICEDUST}}$というGeant4ベースのフレームワークを用いて、シミュレーションを行い、要求されている運動量分解能を満たす磁場測定精度を決定する。
%
\noindent
\uline{\textbf{・本番環境での磁場測定(採用まで)}}\\
 %申請者は2025年に行われるDSおよびCDCなど検出器のインストールを主導し、実験開始の準備を着実に進める。\\
信号電子の運動量$P$は磁場$B$と曲率半径$\rho$より、$0.3B\rho$で計算される。
従って、運動量によって信号電子と背景事象を判別するCOMETでは、磁場の情報はトラッキングアルゴリズムをはじめとする物理解析に使用するため、\textbf{非常に重要である。}
申請者はCDCに必要な運動量分解能を実現する上で許容される磁場の誤差や要求される精度を、ICEDUST$^{\cite{ICEDUST}}$というフレームワーク(COMET実験グループ内で独自に開発された)を用いて決定している。
この結果に従って要求精度を十分満たす磁場測定システムを開発し、インストール後の2025年末に予定されている本番環境での磁場測定を行う。
ここでは、物理測定時の設定磁場である$1\;\rm{T}$での測定に加え、運動量較正(後述)用の低い磁場(約0.5\;\rm{T})での測定も加えて行う。
この測定システムの開発には申請者が修士課程1年時に行ったDS納品後の健全性確認のための磁場測定試験で得られた経験が活きると考えている。この経験から得られたフィードバックを活かし、系統的な誤差や揺らぎを最小限に抑え、$\mathbf{200\;\rm{\textbf{keV/c}}}$\textbf{の運動量分解能}達成に向けた測定システムを開発する。
\\
\noindent
\uline{\textbf{・CDCの運動量較正および宇宙線Run(1年目)}}\\
 2026年度はじめには全ての磁石および検出器の実験室へのインストールが完了している予定である。
そこで、ビーム供給の前に宇宙線を用いた検出器のテストRunを行う。
この測定では、$2\;\rm{GeV}$程度の信号電子やDIO電子と比べて大きなエネルギーを持つ宇宙線の直線的な飛跡が設計通りの位置分解能($200\;\rm{\mu m}$)で再構成できることを確認する。
申請者はデータ取得系のエラーハンドリングを行い、CTHをはじめとするトリガーシステムやCDCの読み出しなど測定系の動作試験に尽力する。
宇宙線試験では、測定系の動作試験に加えて、宇宙線背景事象の見積もりも行う。
偶発的にトリガー条件を満たした宇宙線による背景事象の実験的な観測を行う。
%この測定ではハード、ソフト両面での様々な障害が発生することが予想され、各障害の原因特定およびデバッグ作業を行う。\\
以上の宇宙線試験により検出器やデータ測定システムの準備を完了し、2026年末に予定される低強度でのビーム供給(エンジニアリングラン)に臨む。
%$\pi$から$e$へのビーム即発背景事象(運動量:$70\;\rm{MeV/c}$)と標準模型内でのミューオンの電子への崩壊(運動量:$52.5\;\rm{MeV/c}$)を用いて、検出器の運動量較正を行う。
%信号電子(運動量:$105\;\rm{MeV/c}$)の磁場の強度では飛跡が長くなるため、これらの比較的低運動量電子でも信号電子と同程度の半径の螺旋運動になるように磁場強度を調整する。
%その後、同じ条件でトリガーをかけ、CDCでの飛跡検出を行い、運動量較正を行う。
\\
\noindent
\uline{\textbf{・エンジニアリングランでの運動量較正および物理測定(2年目)}}\\
2026年末にCOMET Phase-Iの1/10の強度でビームでエンジニアリングランを行い、物理測定を行う。
信号電子の運動量測定の前に検出器の運動量較正が必要である。運動量較正には運動量$52.5\;\rm{MeV/c}$のミッシェル崩壊($\mu \rightarrow e \bar{\nu}\nu$)を用いる。
信号電子の運動量と比べてミッシェル崩壊電子は運動量が約半分であるため、信号電子の場合と同程度の螺旋運動の半径にするために磁場強度を半分の$0.5\;\rm{T}$にして運動量較正データ取得を行う。
磁場を変えるため、飛跡の再構成手法も併せて変える必要がある。
磁場強度を変えることで変化する項目としては、磁場の分布(強度に対して線形ではない)、CDC内で電子によりイオン化された負電荷の飛跡などである。
これらの項目に関して採用までの期間で測定した$0.5\;\rm{T}$の磁場マップを用いて、トラッキングアルゴリズムを調整を行い、正確な運動量較正に貢献する。
\\
\noindent
\uline{\textbf{・エンジニアリングランの物理解析およびトラッキングアルゴリズム開発(3年目)}}\\
 2年目の運動量較正の後、定格磁場でのエンジニアリングランおよびその物理解析を行う。
採用前から2年目までに培ったトラッキングに関する理解を元に低強度ビームでの信号電子のアクセプタンスを評価する。
その後、アクセプタンス向上に向けて新たなトラッキングアルゴリズムを開発する。

現在、検出器内で複数周螺旋運動をする\textbf{multiple turn tracks}によるアクセプタンスの低下が問題視されている。
実際に電子の螺旋運動を用いて運動量測定を行うMEG実験$^{\cite{MEG}}$でもこの事象は問題視されており、各周回を判別できるようになったことで\textbf{約}$\mathbf{4\;\rm{\textbf{\%}}}$検出効率がアップしたことが報告されている。
しかし、用いるソレノイド磁場の形状が異なるため、COMET実験でも同じ方法を踏襲することはできず、COMET実験独自のアルゴリズムの開発が要求されている。
実際のDSのソレノイド磁場は鉄ヨークに格納されていても完全に一様ではなく、円筒中心軸においてシミュレーション上で磁石中心に対し検出器両端(磁石中心から$1\;\rm{m}$)で\textbf{約}$\mathbf{15\;\rm{\textbf{\%}}}$磁場が低下する。
即ち、$105\;\rm{MeV/c}$の運動量を持った電子では、磁石中心と比較して検出器両端で螺旋運動の半径が\textbf{約}$\mathbf{7\;\rm{\textbf{cm}}}$大きい飛跡を描く。
multiple turn tracksを残すような電子の平均的な円筒軸方向の運動量は約$1\;\rm{MeV/c}$と予想されており、この運動量では螺旋運動一周の間隔は約$2\;{m}$となる。
従って、周回ごとに数cm単位で曲率半径が変化することが予想され、これを利用したmultiple turn tracksの各周回を分離するアルゴリズムの開発を目指す。
エンジニアリングランでのアクセプタンスの評価、新たなトラッキング手法の開発をPhase-Iにおける世界最高感度でのミューオン電子転換過程探索へと繋げる。

%CDCの1セルが$0.16\;\rm{cm}$(%%多分表現正しくないのでSunさんとかに確認します)であることより、
%飛跡の曲率半径の違いからmultiple turn tracksの各周回を分離するアルゴリズムを開発する。(%%何%感度が良くなるとか、逆に非一様性のせいでTrack Findingに逆効果があるとかはまだ考えれてないです)


%2028年には予定強度のビームでCOMET Phase-Iの測定が開始するため、2年目までに深めた検出器への理解や低強度ビーム試験時で得たトラブルシューティングの経験を元にこの測定を主導する。\\
% 2年目に開発したトラッキングアルゴリズムで実際に得られた物理データの解析を行い、飛跡の再構成の効率やアクセプタンスなどアルゴリズムの性能を評価する。
%低強度ビーム試験データを使ったアルゴリズムのブラッシュアップは2028年以降のCOMET Phase-Iでのビーム測定データ解析、ひいてはミューオン電子転換過程発見に繋がる。
%(%%コメントもらってたと思うのですが、2026年度のデータを解析し尽くすというところでもう少し相談したいです)

\mysubsection{3.研究の特色、独創的な点}
先行実験の SINDRUM-I\hspace{-1.2pt}I$^{\cite{SINDRUM}}$は連続的なDCビームを使用していたことでビームの強度に対して探索可能な感度に制限があった。
一方、COMETでは世界最高のextinction(時間的純度)のパルスミューオンビームを用い、かつ効率の良い背景事象除去が可能なトリガーシステムを用いることで100倍感度を改善できる。
本研究では、未だ同グループでは着手していない\textbf{ソレノイド磁場の非一様性を利用したトラッキングアルゴリズム開発}を行う。
%また、COMETは本研究におけるPhase-Iからアップグレードを見据えており、さらに100倍感度を更新して、現在の上限値の$10^4$倍の感度で探索を行う予定である。
また、関連実験として同じミューオンcLFVである$\mu \rightarrow e + \gamma$を探索するMEG実験$^{\cite{MEG}}$がある。ガンマを放出するか否かの分岐比は理論依存が大きいため、本
研究の結果と組み合わせることで、より詳細な新物理の検証が可能となる。
\mysubsection{4.申請者が担当する部分 5.異なる研究機関での研究従事}
申請者は修士課程でDSの磁場測定とその解析を担っており、現在進行中の実験室環境での磁場測定システム開発にも参画している。
物理解析に用いる磁場測定データの取得を主導し、その経験を活かして、運動量較正をはじめ、物理解析や新たなトラッキング手法開発に貢献する。
また、宇宙線試験から測定終了まで現地に常駐し、J-PARC技術者やコラボレーターらとCOMET実験を支える。
%\mysubsection{5.異なる研究機関での研究従事}
\small
\begin{thebibliography}{99}
	\bibitem{SINDRUM} SINDRUM-II Collaboration, Eur. Phys. J.C47 337-346 (2006)
	\bibitem{ICEDUST}
  R. Derveni, 
  \emph{Comparative analyses of sub-GeV physics in simulations and data for the COMET experiment},  
  Ph.D. thesis, Imperial College London, April 2024.  
  %\href{https://doi.org/10.25560/116399}
	{doi:10.25560/116399}
	\bibitem{MEG}
	A.~M.~Baldini \textit{et al.}, ``Search for the lepton flavour violating decay $\mu^+ \to e^+ \gamma$ with the full dataset of the MEG experiment,'' \textit{European Physical Journal C}, vol.~76, no.~8, p.~434, 2016. 
	%\doi{10.1140/epjc/s10052-016-4271-x}
\end{thebibliography}
%s02_purpose_plan_dcpd
%\JSPSInstructions		% <-- 留意事項。これは消すか、コメントアウトしてください。
%begin 研究目的と研究計画shorter ====================
%\textbf{\\     *** 以下は、あくまで例です。真似しないでください。 ***\\
%     *** 本文はもちろん、節の切り方や論理の組み方は   ***\\
%     *** ご自分の気に入ったスタイルで書いてください。  ***}


%end 研究目的と研究計画shorter ====================
\input{pieces/p02_purpose_plan_01}

%#Split: 03_rights  
%#PieceName: p03_rights
\input{pieces/p03_rights_00}
\section{人権の保護及び法令等の遵守への対応}
%    <<最大 1ページ>>

% s09_rights
%begin 人権の保護及び法令等の遵守への対応 ====================
該当なし
	%象の卵のES細胞の培養、象のクローンの生成などは行わない。
%
	%\LaTeX の便利な機能については、\texttt{egg\_***.tex} や\texttt{sample\_pdf/egg\_***.pdf}をご覧ください。
%end 人権の保護及び法令等の遵守への対応 ====================

\input{pieces/p03_rights_01}

%#Split: 04_abilities  
%#PieceName: p04_abilities
% p04_abilities_00.tex
\KLBeginSubjectWithHeaderCommands{04}{4}{【4】研究遂行力の自己分析}{2}{F}{}{\DCPDFirstSubjectPageStyle}{\DCPDDefaultPageStyle}

\section{【4】研究遂行力の自己分析}
%    <<最大 2ページ>>

% s14_abilities
%\SelfReviewInstructions\\% <-- 留意事項:これは消すか、コメントアウトしてください。
\noindent
\textbf{(1) 研究に関する自身の強み}\\
\vspace{-0.5cm}
\mysubsection{3Dプリンターで作成したシンチレーターの性能評価}
学部4年生時に行った上記題目の研究を通じて、\textbf{「手数を惜しまず仮説検証を繰り返す根気強さ}」と\textbf{「得られた結果から新たな仮説を立てる創造性」}を養うことができた。
当研究は先行研究に乏しく、印刷時の印刷パラメーターや作成したシンチレーターの特性などわかっていないことが多かった。
そこで最初に印刷に成功した時のパラメーターをもとに問題点に対して、仮説検証を繰り返した。
具体的に問題点というのは印刷完了後数分放置すると、シンチレーターが白濁し発光量が低下することであった。
この現象に対し、環境紫外線(実験室の蛍光灯)や空気中不純物との表面樹脂の結合が原因ではないかと仮定し調査した。
結論、表面樹脂の空気中不純物との結合が原因であることがわかり、空気との接触を絶った条件では環境紫外線に晒した方が発光量が増加することがわかった。
この結果より、紫外線により不完全な光重合が促進され、凝縮率が上がったため発光量が増したのではないかと考え、印刷後に二次的に紫外線を照射し、照射時間で発光量の違いを調査した。
その結果、照射時間5時間以降発光量が横ばいになり、紫外線照射時間の最適化に成功した。
\mysubsection{COMET実験に用いる検出器ソレノイドの健全性確認}

修士1年次から、COMET Phase-Iに用いる検出器ソレノイド(DS)が納品され、申請者はその磁場測定および健全性確認を行っている。
その磁場測定試験の結果、ソレノイドの円筒軸に沿った磁場分布にシミュレーションと比較して大きな違いが見られた。
測定環境に存在していた鉄柱の影響など様々な測定条件を考慮し、補正する解析を行った。
\begin{wrapfigure}[12]{r}{190pt}{}
	\centering
	 \vspace{-0.4cm}
	 \hspace*{-15pt}
	 \includegraphics[width=7cm]{figs/PeakPositions.png}
	 \vspace{-1.0cm}
	 \caption{\small{円筒軸方向の磁場分布の結果\\
	 }}
	 \label{Fig:peakpositions}
	\end{wrapfigure} 
結果的に個人的にはあまり影響がないと思い込んでいた磁場測定に用いたセンサーのコンマ数度の傾きが最も現象を説明できる要因であることがわかった。
その際に\textbf{現象を体系的に捉える力}とその力を元に\textbf{大小問わず全ての要因を正確に評価する力}が身についた。\\
また、COMET実験のコンセプトではトラッキングと磁場は非常に密接に関わり合っているが、ソフトウェア面からトラッキングに関わる中国のグループと
実際にDSのセットアップを行い、磁場測定を行うグループの間に認識の乖離やコミュニケーションが過疎化していたという事実があった。
その事実に問題を感じた申請者は磁場測定グループの一員として積極的にトラッキンググループとコミュニケーションを取り、互いの理解の深化に努めている。
このような日々の活動に加えてCOMET全体の定期的な会議なども通じて\textbf{日英を問わず、自分の理解を相手に伝える力}が養われたと感じている。

%申請者の強みは日英を問わないコミュニケーション力である。
%COMET配属前の学部4年生の頃に参加した会議では、初対面の研究者らにも積極的に話しかけ、配属前から名前を覚えてもらうなど、良好な関係を築くことができた。
%以降もCOMETの活動を通じて、共同研究者と積極的に交流・議論を重ね、COMET内における学術的・人的ネットワークの広がりと深化に貢献している。
%多くの研究者が協力して一つの実験を遂行するには、良好な人間関係と円滑な意思疎通を基盤とした組織風土が不可欠であり、申請者の人間性とコミュニケーション能力は、そのような環境の持続的な醸成において大きな役割を果たすと考えている。\\
%加えて、申請者の強みは上述のコミュニケーションを英語でも行うことができる点である。
%COMETは国際コラボレーションであるため、英語での積極的なコミュニケーションが求められる。
%各種会議での対面のコミュニケーションに加え、
%申請者は、トラッキングのソフトウェアに関して学習するために中心となっている中国のグループと積極的に交流している。
%自分の理解を深めるだけでなく、トラッキングソフトウェアグループと磁石関連を担うKEK低温セクションのリエゾンのような役割も果たし、物理解析の主軸であるトラッキングソフトウェアとトラッキングに欠かせない磁場測定の現場との情報交換の活性化を担っている。

\mysubsection{研究成果}
\noindent
国際学会(査読あり、審査中):\\
\uline{Shohta Takami}, Masaharu Aoki, Ryo Nagai, Kenichi Sasaki, Naoyuki Sumi, Makoto Yoshida, Masami Iio, Hirokatsu Ohhata, "Magnetic field measurement and analysis of the detector soleonid for the COMET experiment",
International Conference on Magnet Techonology (MT29), Boston, USA, 2025.Jul\\
\noindent
研究発表(査読なし):\\
\uline{高見 翔太}「3Dプリンターで作成したシンチレーターの性能評価」第12回高エネルギー物理春の学校、滋賀県、2024年5月(ポスター発表)\\
\uline{高見 翔太}、青木正治、永井遼、佐々木憲一、角直幸、吉田誠、飯尾雅美、大畠洋克「COMET実験に用いる検出器ソレノイドの磁場測定および健全性確認」、日本物理学会2025年春季大会、18aT1-8、オンライン、2025年3月(口頭発表)

\mysubsection{15年間の競技活動}
申請者は幼少期からサッカーを競技者として行っており、学部生時は学業の傍ら大阪大学体育会サッカー部にて競技活動に勤しんでいた。
当部は関西2部リーグに所属し、入部当時で80人規模のチームであった。全くの無名の高校出身の申請者にとっては挑戦的であったが、誰よりも身を呈して守備をすることと正確なロングキックという自分の武器を磨き続け、Aチームまで上り詰めた。
当部での4年間では競技者としての成長もさることながら人間として大きく成長できたと自負している。
そのように感じる点は大きく二つある。
一つ目は\textbf{粘り強い胆力}である。入部から二年間はチーム内最下層のCチームから昇格することができなかった。同様の状況の同期にはすでに退部している者もおり、諦めるには十分な状況だったと当時を振り返って思う。
しかし、決して事切れることなく、研鑽を続け、三年目でAチームに昇格することができた。辛酸を嘗め続けたことにより、\textbf{目標に向けて絶えず努力を続ける胆力}が身についた。

二つ目は人間力である。
申請者は3年目ではAチームに昇格したものの4年目の最後の半年ではBチームに降格してしまった。
それでも、必ず本来屈辱であるAチームの試合の応援に毎週駆けつけ、Bチームの試合ではキャプテンとしてチームを鼓舞し続けた。
日々の練習では、消灯時間までの自主練習を欠かさなかった。
このような行動を心がけたのは、自分のことだけでなく、組織全体のことまで考えていたからである。結果が出ないからと事切れるのは簡単であるが、全体の士気に大きな影響が出る。
最後の年である4年目に望む結果が出なかったのは非常に屈辱的だったが、最終的に大きく人間的に成長できたと実感している。
特に高エネルギー領域では、色々な機関とのコラボレーションで実験を行うことが多い。
従って個人で完結する研究活動は少なく、実験グループ全体の目的は何であるかということを強く意識して活動することは非常に重要であり、
そのような観点から申請者は実験成功において\textbf{非常にポジティブな人材}であると考える。
(%%しょうもないことなんですけど自分にはこれくらいしかないんでかきました)

%また、COMET以外でも積極的に学術交流を行なっており、KEK\footnote{高エネルギー加速器研究機構}主催の春の学校という若手研究者の集まりでは
%\PapersInstructions\\ %<-- 留意事項:これは消すか、コメントアウトしてください。
%begin 自己分析 ====================
	
	%\begin{enumerate}
	%	
	%	% 下のように書いてもいいけど、めんどくさいし、表示の仕方を変えようとしたら大変。
	%	\item ``Egg of Elephant-Bird'', 
	%			\underline{A.~Cooper},
	%			Nature, {\bf 409}, 704-707 (2001).	% 	
	%	\input{jack_pub}	% << only for demonstration. Please delete it or comment it out.
	%\end{enumerate}
%end 自己分析 ====================

\vspace{5mm}
\noindent
\textbf{(2) 今後研究者として更なる発展のため必要と考えている要素}
%begin 今後必要な要素 ====================
\mysubsection{ソフトウェア能力}
2026年以降に計画されている物理測定データの解析には高度なプログラミングスキルが必要となる。
高エネルギー物理学の領域にはGeant4(素粒子の相互作用シミュレーション)をはじめとする解析フレームワークが多くあり、ソフトウェアに対する深い理解が要求される。
実際にCOMET実験グループでは、ICEDUSTというGeant4ベースの荷電粒子のトラッキングフレームワークが独自に開発されている。
加えて、最近では機械学習を用いた解析も盛んに行われおり、ソフトウェアに関する幅広い知識と経験を持つことは解析の技術という観点で非常に重要である。
申請者は今後ICEDUSTをはじめとするソフトウェアの学習と実践に励み、電子のトラッキングに貢献する。
\mysubsection{国際的なコミュニケーション能力}
本研究は国際共同実験であるため、国外の共同研究者との英語でのコミュニケーションが求められる。
申請者は高校時代までの国際経験を活かし、COMETグループの会議等を通じて、積極的に国外の研究者とコミュニケーションをとるよう努めている。
特にトラッキングを主導する中国IHEPの研究者らとは密にコミュニケーションをとっており、ソフトウェアの知識を学習することに加えて磁場測定に係る現場の情報を共有するなど
コラボレーションがより活発になるよう心掛けている。
また、日本で行われたMu2eグループとの会議に出席し、磁場測定システムの課題や実情に関して深く議論することで本番環境の磁場測定システムに対する解像度を高めることができた。
今後もより活発に様々な研究者とより密に議論できるようになっていきたいと考えている。
\mysubsection{理論に対する深い理解}
申請者は学部四年生から素粒子物理実験の研究室に所属しており、素粒子物理学における理論のゼミを授業の枠以外にも自主的に同輩らと行っている。
しかし、素粒子物理の世界は奥が深く、到底十分な理解できているとは考えていない。今後物理解析などを通じてより新物理の本質に迫っていこうとする上で、
創造性あふれるアイデアは根底の理論に対する深い理解から生まれると考える。
従って、申請者は今後も理論の理解を深めていき、クリティカルなアイデアを生み出せるようになりたい。
\mysubsection{創造性ゆたかな発想}
研究者として実験を行う上で、固定概念にとらわれない解析法の発想や素朴な好奇心は非常に重要であると考えている。
実際、日々研究をともに行う教授陣やポスドクらはおしなべてこの二つの要素を持っていると感じる。
それは、研究の中で生じる大小様々な疑問を突き詰めてきた結果であり、申請者自身もそのような研究者へと成長していきたいと強く思う。

%end 今後必要な要素 ====================

\input{pieces/p04_abilities_01}

%#Split: 99_tail
\input{pieces/hook9} % pieces
\end{document}

